%%%%%%%%%%%%%%%%%%%%%%%%%%%%%%%%%%%%%%%%%
% "ModernCV" CV and Cover Letter
% LaTeX Template
% Version 1.3 (29/10/16)
%
% This template has been downloaded from:
% http://www.LaTeXTemplates.com
%
% Original author:
% Xavier Danaux (xdanaux@gmail.com) with modifications by:
% Vel (vel@latextemplates.com)
%
% License:
% CC BY-NC-SA 3.0 (http://creativecommons.org/licenses/by-nc-sa/3.0/)
%
% Important note:
% This template requires the moderncv.cls and .sty files to be in the same 
% directory as this .tex file. These files provide the resume style and themes 
% used for structuring the document.
%
%%%%%%%%%%%%%%%%%%%%%%%%%%%%%%%%%%%%%%%%%

%----------------------------------------------------------------------------------------
%	PACKAGES AND OTHER DOCUMENT CONFIGURATIONS
%----------------------------------------------------------------------------------------

\documentclass[11pt,a4paper,sans]{moderncv} % Font sizes: 10, 11, or 12; paper sizes: a4paper, letterpaper, a5paper, legalpaper, executivepaper or landscape; font families: sans or roman
\usepackage[left=15mm, right=15mm, top=15mm, bottom=25mm]{geometry}


\moderncvstyle{casual} % CV theme - options include: 'casual' (default), 'classic', 'oldstyle' and 'banking'
\moderncvcolor{blue} % CV color - options include: 'blue' (default), 'orange', 'green', 'red', 'purple', 'grey' and 'black'

\usepackage{lipsum} % Used for inserting dummy 'Lorem ipsum' text into the template
\usepackage{fancyhdr}
\usepackage[polish]{babel}
\usepackage[utf8]{inputenc}
\usepackage{polski}
\usepackage[T1]{fontenc}
\frenchspacing
%\usepackage[scale=0.75]{geometry} % Reduce document margins
%\setlength{\hintscolumnwidth}{3cm} % Uncomment to change the width of the dates column
%\setlength{\makecvtitlenamewidth}{10cm} % For the 'classic' style, uncomment to adjust the width of the space allocated to your name

%----------------------------------------------------------------------------------------
%	NAME AND CONTACT INFORMATION SECTION
%----------------------------------------------------------------------------------------

\firstname{Konrad} % Your first name
\familyname{Traczyk} % Your last name

% All information in this block is optional, comment out any lines you don't need
\title{Embedded Software Developer}
\address{ul. Egejska 13/10}{Warszawa, 02-764}
\mobile{(+48) 691-494-830}
\email{konrad.traczyk122@gmail.com}
%\homepage{} % The first argument is the url for the clickable link, the second argument is the url displayed in the template - this allows special characters to be displayed such as the tilde in this example

\photo[70pt][0.4pt]{pictures/Zdjecie_KT.JPG} % The first bracket is the picture height, the second is the thickness of the frame around the picture (0pt for no frame)
%\quote{"A witty and playful quotation" - John Smith}

\setlength{\footskip}{20pt}
\extrainfo{ %
\renewcommand{\arraystretch}{0.5}
\begin{tabular}{@{\hspace{2em}}c@{\hspace{2em}}}
\tiny Wyrażam zgodę na przetwarzanie moich danych osobowych dla potrzeb niezbędnych do realizacji procesu rekrutacji  \\
\tiny  (zgodnie z Ustawą z dnia 29.08.1997 roku o Ochronie Danych Osobowych; tekst jednolity: Dz. U. 2016 r. poz. 922).
\end{tabular}
} 
%\fancyfoot[l]{\parbox[b]{15cm}{\tiny Niniejszym oświadczam, że wyrażam zgodę na przetwarzanie moich danych osobowych w celu przeprowadzenia procesu rekrutacji (zgodnie z ustawą z dnia 29.08.1997r. o ochronie danych osobowych Dz. U. Nr 133, poz. 883).}}

%----------------------------------------------------------------------------------------

\begin{document}

%----------------------------------------------------------------------------------------
%	COVER LETTER
%----------------------------------------------------------------------------------------

% To remove the cover letter, comment out this entire block

%\clearpage

%\recipient{HR Department}{Corporation\\123 Pleasant Lane\\12345 City, State} % Letter recipient
%\date{\today} % Letter date
%\opening{Dear Sir or Madam,} % Opening greeting
%\closing{Sincerely yours,} % Closing phrase
%\enclosure[Attached]{curriculum vit\ae{}} % List of enclosed documents

%\makelettertitle % Print letter title

%\lipsum[1-2] % Dummy text
%\lipsum[4] % Dummy text

%\makeletterclosing % Print letter signature

%\newpage

%----------------------------------------------------------------------------------------
%	CURRICULUM VITAE
%----------------------------------------------------------------------------------------

\makecvtitle % Print the CV title

%----------------------------------------------------------------------------------------
%	EDUCATION SECTION
%----------------------------------------------------------------------------------------
%----------------------------------------------------------------------------------------
%	WORK EXPERIENCE SECTION
%----------------------------------------------------------------------------------------

\section{Doświadczenie zawodowe}

\cventry{09.2017 -- Obecnie}{Embedded Software Developer}{\textit{Mudita Sp. z o.o}}{Warszawa}{}{Developer w zespole wdrażającym system operacyjny dla bioneutralnego telefonu komórkowego. Odpowiedzialny za budowę modułowego, rozproszonego symulatora urządzenia oraz oprogramowanie mikrokontrolerów. Praca w języku C oraz C++ na systemie operacyjnym Arch Linux, członek zespołu scrumowyego.}

%------------------------------------------------

\cventry{08.2016 -- 09.2017}{Junior Software Engineer}{\textit{Samsung Electronics Polska}}{Warszawa}{}{Pracownik działu Visual Display zajmujący się wdrażaniem i utrzymywaniem funkcjonalności telewizji hybrydowej w telewizorach Smart TV. Praca w językach C++ oraz Java Script na systemie operacyjnym Linux.}

%------------------------------------------------

\cventry{02.2015 -- 08.2016}{Embedded Software Developer}{\textit{LogicIT Mateusz Brzozowski}}{Warszawa}{}{Udział w realizacji projektu OneMeter - urządzenia do zdalnego odczytu danych z liczników energii elektrycznej. Odpowiedzialny za implementację oprogramowania bare metal w języku C dla urządzenia z wykorzystaniem protokołu Bluetooth Low Energy.}

\cventry{08.2014 -- 09.2014}{Praktyki zawodowe}{\textit{Przemysłowy Instytut Automatyki i Pomiarów}}{Warszawa}{}{Praktyki na stanowisku Programisty Systemów Wbudowanych. Praca z mikrokontrolerami w robotach mobilnych.}


\section{Wykształcenie}

\cventry{2016--2018}{Studia II stopnia}{Wydział Mechatroniki}{Politechnika Warszawska}{\newline\textit{Kierunek: Automatyka i Robotyka}}{\newline\textit{Specjalizacja: Informatyka Przemysłowa.}} % Arguments not required can be left empty

\cventry{2012--2016}{Studia I stopnia}{Wydział Mechatroniki}{Politechnika Warszawska}{\newline\textit{Kierunek: Automatyka i Robotyka}}{\newline\textit{Specjalizacja: Robotyka.}} % Arguments not required can be left empty


%----------------------------------------------------------------------------------------
%	COMPUTER SKILLS SECTION
%----------------------------------------------------------------------------------------

\section{Technologie}

\cvitem{Podstawowe}{\textbf{JavaScript, Java, SQL}, \textit{HTTP, libwebsockets, OpenCV, UDP, GSM, GPS}, \LaTeX}
\cvitem{Średniozaawansowane}{\textit{TCP/IP, Bluetooth Low Energy}, Perforce, GDB, Linux (programista i użytkownik), Microsoft Windows (użytkownik)}
\cvitem{Zaawansowane}{\textbf{C, C++, Qt, ARM CortexM}, Git}


\section{Umiejętności}
\begin{itemize}
	\item Umiejętność projektowania podstawowych układów w programie Altium Designer
	\item Doświadczenie z oscyloskopem i analizatorem stanów logicznych
	\item Czytanie schematów elektronicznych
	\item Doświadczenie w pracy w zespole scrumowym oraz z wykorzystaniem systemu Jira
\end{itemize}


\pagebreak

\section{Prace dyplomowe i projekty}

\cvitem{Praca Magisterska}{\emph{Projekt urządzenia do lokalizacji pojazdów w trybie on i offline}}
\cvitem{Opis}{Projekt oraz oprogramowanie do urządzenia elektronicznego służącego do lokalizacji pojazdu oraz zdalnego powiadamiania w razie jego kradzieży. Urządzenie pozwala dodatkowo na dynamiczną analizę stylu jazdy kierowcy. Dane są cyklicznie gromadzone w bazie danych obsługiwanej przez zaimplementowaną w ramach pracy aplikację serwerową napisaną w języku C++. Ich wizualizacja jest dokonywana na dedykowanej stronie internetowej przy użyciu języka Java Script oraz Google Maps API. \newline
\textbf{Technologie embedded:} \textit{C, Bluetooth Low Energy, NFC, GSM, GPS, AES128 \newline}\textbf{Technologie serwerowe:} \textit{C++, Qt, HTTP, SQL \newline}\textbf{Technologie webowe:} \textit{Java Script, HTML, CSS, Google Maps, HTTP}}

\cvitem{Praca inżynierska}{\emph{Projekt personalnego urządzenia śledzącego pozycję geograficzną sprzężonego z
aplikacją mobilną}}
\cvitem{Opis}{Projekt i oprogramowanie urządzenia służącego do lokalizacji osób, zapisywania przebytych tras i ich wizualizacji na telefonie z systemem Android z wykorzystaniem biblioteki Google Maps. \newline \textbf{Technologie: }\textit{C/Bluetooth Low Energy/GPS/Android/Java}}

\cvitem{Projekt studencki}{\emph{Program do rozpoznawania i klasyfikacji obrazów w oparciu o własną implementację
sieci neuronowej \newline \textbf{Technologie: }C++/OpenCV/Qt}}

\cvitem{Projekt studencki}{\emph{Odtwarzacz plików .wav z karty SD w oparciu o mikrokontroler STM32F407 \newline \textbf{Technologie: }C/FATFS/Elektronika}}

%----------------------------------------------------------------------------------------
%	AWARDS SECTION
%----------------------------------------------------------------------------------------

\section{Osiągnięcia}

\cvitem{12.2016}{Nagroda Best Employee w dziale Visual Display Samsung Electronics Polska}
\cvitem{10.2015}{Udział w konferencji Global Tech Tour organizowanej przez Nordic Semiconductor, dotyczącej mikrokontrolera nRF52832.}
\cvitem{2013-2015}{Stypendium za wyniki w nauce dla 10\% studentów na kierunku mechatronika w latach}



%----------------------------------------------------------------------------------------
%	COMMUNICATION SKILLS SECTION
%----------------------------------------------------------------------------------------

%----------------------------------------------------------------------------------------
%	LANGUAGES SECTION
%----------------------------------------------------------------------------------------

\section{Języki}

\cvitemwithcomment{polski}{język ojczysty}{}
\cvitemwithcomment{angielski}{średniozaawansowany}{}
\cvitemwithcomment{niemiecki}{podstawowy}{}

%----------------------------------------------------------------------------------------
%	INTERESTS SECTION
%----------------------------------------------------------------------------------------

\section{Zainteresowania}

\renewcommand{\listitemsymbol}{-~} % Changes the symbol used for lists

\cvlistdoubleitem{Elektronika}{Programowanie}
\cvlistdoubleitem{Systemy wbudowane}{Internet of Things}
\cvlistdoubleitem{Motoryzacja}{Automotive}


%----------------------------------------------------------------------------------------

\end{document}